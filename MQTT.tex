\documentclass[12pt]{article}

\usepackage{sbc-template}

\usepackage{graphicx,url}

\usepackage[brazil]{babel}   
%\usepackage[latin1]{inputenc}  
\usepackage[utf8]{inputenc}  
% UTF-8 encoding is recommended by ShareLaTex

     
\sloppy

\title{Smart Class: MQTT}

\author{Bruno R. Scherer\inst{1}, Bruno Seiji\inst{1}, Felipe H. B. Bastos\inst{1}, Girlian Gildo\inst{1}, Jamelly F. Ferreira\inst{1}, Victoria Beatriz\inst{1} }


\address{Faculdade de Engenharia da Computação e Telecomunicações \\-- Universidade Federal do Pará
  (UFPA)\\
  Caixa Postal 479 -- 66.075-110 -- Belém -- PA -- Brazil
  \email{\felipe@itec.ufpa.br,}
}

\begin{document} 

\maketitle

\begin{abstract}
  

  
  
\end{abstract}
     

\section{Introduction}



\section{Message Queuing Telemetry Transport protocol} \label{sec:sdn}
\subsection{Overview}



\begin{figure}[ht]
\centering
\includegraphics[width=.7\textwidth]{figure1.png}
\caption{Components of SDN}
\label{figure1}
\end{figure}

\subsection{NFV}

The NV (Network Virtualization) is a newly concept about networks using cloud computing and its resources. With the cloud computing, it became possible to set and use resources distributed in the network in according with the needs of the user. So, the resource is distributed and the control about this is remote and can be more precise and with higher performance than traditional architecture.

With the same idea of NV, emerged the concept o NFV (Network Function Virtualization), this concept is about to emulate into computational resources the elements of the network, like routers, firewalls and EPCs. This is very important to make networks more flexible, scalable and simple to manage.

\section{MQTT brokers}

\section{MQTT usage: NODE.JS case}

\section{MQTT herdware usage: ESP8266 case}

\subsection{Software design}

\subsection{Hardware design}

\section{Simulation analizies}

\section{Results and discussions}

\section{Conclusions}



\section{References}
      Hamid F, HyunYong L and Akihiro N, ’Software-Defined Networking: A survey’, Computer Networks, 2015.

      Garzon J, Adamuz-Hinojosa O, Ameigeiras P, ’Handover Implementation in a 5G SDN-based Mobile Network Architecture’, PIMRC: Mobile and Wireless Networks, 2016

      Akram H, Aniruddha G, ’Software-Defined Networking: Challenges and research opportunities for Future Internet’, Computer Networks, 2015.

      Nunes B, Marc M,’A Survey of Software-Defined Networking: Past, Present, and Future of Programmable Networks’, IEEE Communications surveys and tutorials, vol. 16, No. 3, Third quarter 2014.

\end{document}
